% \section{Data Sub-Packet} \label{section:dataSub}
%
% The data sub-packet consists of 32 "chunks" (30 sensors and 2 MAC addresses).  Each "chunk" follows one of seven formats.\\

%%%%%%%%%%%%%%%%%%%%
\newpage
\section{Data Formats}

The data sent in each sub-packet is encoded in one or more formats. Currently
we define eight formats for various types of data including integers, bytes,
and floating point numbers. The numerical range of these representations is
restricted to within the bounds of values that we expect from the various sensors
and other sources. Thus the encoding schemes are specifically designed to
effectively and efficiently encode the values expected in the sensor streams.
The eight formats, and the encoding schemes are listed below.

\begin{table}[H]
    \centering
    {\rowcolors{2}{black!8}{black!2}
    \begin{tabular}{|c|c|c|c|}
        \hline
        \textbf{Format} & \textbf{Number of Bytes Used} & \textbf{Value Represented} & \textbf{Value Range} \\
%          & \textbf{Encoding}
        \hline
        \hline
        1 & 2 & unsigned int\_16 input & 0 -- 65535 \\ %& MSByte LSByte\\
        2 & 2 & int\_16 input & $\pm$\{0 -- 32767\} \\%& [1Sign|7-MSBits] LSByte \\
        3 & 6 & byte input[6] & 0x00 -- 0xffffffffffff \\%& MSByte1 MSByte2 MSByte3 MSByte4 MSByte5 LSByte \\
        4 & 3 & unsigned long\_24 input & 0 -- 16777215 \\%& MSByte1 MSByte2 LSByte \\
        5 & 3 & long\_24 input & $\pm$\{0 -- 8388607\} \\%& [1Sign|7-MSBits] MSByte2 LByte \\
        6 & 2 & float input & $\pm$\{0 -- 127.99\} \\%& [1Sign|7Bits\_Int] [0|7Bits\_Frac]\\
        7 & 4 & byte input[4] & 0x00 -- 0xffffffff \\%& MSByte1 MSByte2 LSByte2 LSByte\\
        8 & 2 & float input & $\pm$\{0 -- 31.999\} \\%& [1Sign|5Bits\_Int|2MSBits\_Frac]  8LSBits\_Frac\\
        \hline
    \end{tabular}
    }
    \caption{Data formats}
    \label{table:overall}
\end{table}

%%%%%%%%%%%%%%%%%%%%

\subsection{Format 1}

This 2 byte format is used to transmit an integer between 0 and 65535. The
number is split and serialized as follows --\\

\begin{table}[H]
\centering
\begin{tabular}{|c|c|}
\hline
% column1a & column2a \\
\noalign{\hrule height 2pt}
\multicolumn{1}{!{\vrule width 2pt}c!{\vrule width 1pt}}{8 Most Significant Bits} &
  \multicolumn{1}{!{\vrule width 1pt}c!{\vrule width 2pt}}{8 Least Significant Bits} \\
\noalign{\hrule height 2pt}
Byte[0] & Byte[1] \\
\hline
\end{tabular}
\end{table}


\subsection{Format 2}
This 2 byte format is used to transmit an integer between -32767 and 32767. The
number is split and serialized as follows --\\

\begin{table}[H]
\centering
\begin{tabular}{|c|c|}
\hline
% column1a & column2a \\
\noalign{\hrule height 2pt}
\multicolumn{1}{!{\vrule width 2pt}c!{\vrule width 1pt}}{Sign Bit | 7 Most Significant Bits} &
\multicolumn{1}{!{\vrule width 1pt}c!{\vrule width 2pt}}{8 Least Significant Bits} \\
\noalign{\hrule height 2pt}
Byte[0] & Byte[1] \\
\hline
\end{tabular}
\end{table}

The Sign Bit which is the most significant bit in Byte 0 is set as follows ---

\begin{table}[H]
\centering
\begin{tabular}{|c|c|}
\hline
Positive Integer & 0 \\ 
\hline
Negative Integer & 1 \\
\hline
\end{tabular}
\end{table}



\subsection{Format 3}

This 6 byte format is used to transmit an array of 6 bytes. The array is serialized as follows --\\

\begin{table}[H]
\centering
\begin{tabular}{|c|c|c|c|c|c|}
\hline
% column1a & column2a \\
\noalign{\hrule height 2pt}
\multicolumn{1}{!{\vrule width 2pt}c!{\vrule width 1pt}}{Array[0]} &
\multicolumn{1}{!{\vrule width 2pt}c!{\vrule width 1pt}}{Array[1]} &
\multicolumn{1}{!{\vrule width 2pt}c!{\vrule width 1pt}}{Array[2]} &
\multicolumn{1}{!{\vrule width 2pt}c!{\vrule width 1pt}}{Array[3]} &
\multicolumn{1}{!{\vrule width 2pt}c!{\vrule width 1pt}}{Array[4]} &
\multicolumn{1}{!{\vrule width 1pt}c!{\vrule width 2pt}}{Array[5]} \\
\noalign{\hrule height 2pt}
1st Byte & 2nd Byte & 3rd Byte & 4th Byte & 5th Byte & 6th Byte \\
\hline
\end{tabular}
\end{table}

\subsection{Format 4}

This 3 byte format is used to transmit an integer between 0 and 16777215. The
number is split and serialized as follows --\\

\begin{table}[H]
\centering
\begin{tabular}{|c|c|c|}
\hline
% column1a & column2a \\
\noalign{\hrule height 2pt}
\multicolumn{1}{!{\vrule width 2pt}c!{\vrule width 1pt}}{8 Most Significant Bits} &
\multicolumn{1}{!{\vrule width 2pt}c!{\vrule width 1pt}}{Bits 15 -- 8 } &
  \multicolumn{1}{!{\vrule width 1pt}c!{\vrule width 2pt}}{8 Least Significant Bits} \\
\noalign{\hrule height 2pt}
Byte 0 & Byte 1 & Byte 2\\
\hline
\end{tabular}
\end{table}

\subsection{Format 5}

This 3 byte format is used to transmit an integer between -8388607 and 8388607. The
number is split and serialized as follows --\\

\begin{table}[H]
\centering
\begin{tabular}{|c|c|c|}
\hline
% column1a & column2a \\
\noalign{\hrule height 2pt}
\multicolumn{1}{!{\vrule width 2pt}c!{\vrule width 1pt}}{Sign Bit | 7 Most Significant Bits} &
\multicolumn{1}{!{\vrule width 2pt}c!{\vrule width 1pt}}{Bits 15 -- 8 } &
  \multicolumn{1}{!{\vrule width 1pt}c!{\vrule width 2pt}}{8 Least Significant Bits} \\
\noalign{\hrule height 2pt}
Byte 0 & Byte 1 & Byte 2\\
\hline
\end{tabular}
\end{table}

The Sign Bit which is the most significant bit in Byte 0 is set as follows ---

\begin{table}[H]
\centering
\begin{tabular}{|c|c|}
\hline
Positive Integer & 0 \\
\hline
Negative Integer & 1 \\
\hline
\end{tabular}
\end{table}



\subsection{Format 6}

This 2 byte format is used to transmit a floating point number between
-127.99 and 127.99. Only 2 fractional places are allowed in this format and
the number is serialized as follows --\\

\begin{table}[H]
\centering
\begin{tabular}{|c|c|c|c|c|c|}
\hline
% column1a & column2a \\
\noalign{\hrule height 2pt}
\multicolumn{1}{!{\vrule width 2pt}c!{\vrule width 1pt}}{Sign Bit| 7 bit representation of Integer part} &
\multicolumn{1}{!{\vrule width 2pt}c!{\vrule width 1pt}}{ 0 | 7 bit representation of the Fractional part} \\
\noalign{\hrule height 2pt}
Byte 0 & Byte 1\\
\hline
\end{tabular}
\end{table}

As shown above, the leading bit of the Byte 1 is always set to 0, and
the Sign Bit which is the most significant bit of Byte 0 is set as follows ---

\begin{table}[H]
\centering
\begin{tabular}{|c|c|}
\hline
Positive Number & 0 \\
\hline
Negative Number & 1 \\
\hline
\end{tabular}
\end{table}


\subsection{Format 7}

This 4 byte format is used to transmit an array of 4 bytes. The array is serialized as follows --\\

\begin{table}[H]
\centering
\begin{tabular}{|c|c|c|c|c|c|}
\hline
% column1a & column2a \\
\noalign{\hrule height 2pt}
\multicolumn{1}{!{\vrule width 2pt}c!{\vrule width 1pt}}{Array[0]} &
\multicolumn{1}{!{\vrule width 2pt}c!{\vrule width 1pt}}{Array[1]} &
\multicolumn{1}{!{\vrule width 2pt}c!{\vrule width 1pt}}{Array[2]} &
\multicolumn{1}{!{\vrule width 2pt}c!{\vrule width 1pt}}{Array[3]}\\
\noalign{\hrule height 2pt}
Byte 0 & Byte 1 & Byte 2 & Byte 3 \\
\hline
\end{tabular}
\end{table}


\subsection{Format 8}

This 2 byte format is used to transmit a floating point number between
-31.999 and 31.99. Only 3 fractional places are allowed in this format and
the number is serialized as follows --\\

\begin{table}[H]
\centering
\begin{tabular}{|c|c|c|c|c|c|}
\hline
% column1a & column2a \\
\noalign{\hrule height 2pt}
\multicolumn{1}{!{\vrule width 2pt}c!{\vrule width 1pt}}{Sign Bit| 5 bit representation of Integer | 2 most significant bits of fraction} &
\multicolumn{1}{!{\vrule width 2pt}c!{\vrule width 1pt}}{ 8 least significant bits of the fraction} \\
\noalign{\hrule height 2pt}
Byte 0 & Byte 1\\
\hline
\end{tabular}
\end{table}

As shown above, the format uses 5 bits for representing the integer part and 10 bits to represent the
fractional part. The Sign Bit which is the most significant bit of Byte 0 is set as follows ---

\begin{table}[H]
\centering
\begin{tabular}{|c|c|}
\hline
Positive Number & 0 \\
\hline
Negative Number & 1 \\
\hline
\end{tabular}
\end{table}

