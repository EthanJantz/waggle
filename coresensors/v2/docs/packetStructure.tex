\newpage
\section{Data Transmission} \label{section:overall}

The data from the sensor boards is sent as a formatted unit of data -- a transmission
packet. A transmission packet is composed of several data sub-packets, each
of which carries information pertaining to the parameter.
The transmission packet format and the data sub-packets are described here.

\subsection{Transmission Packet}
A transmission packet can be separated into 6 segments.
The structure of the transmission packet relies on positions of Bytes and predefined values for those Byte segments. 
Table 1 below illustrates how the segments are organized in a transmission packet.
\\

\begin{table}[h!]
    \centering
    \caption{Transmission Packet structure}
    \label{table:tran}
    \begin{tabular}{|c|c|c|c|c|c|}
        \hline
        \rowcolor{black!8}
        \textbf{Preamble} & \textbf{Seq. | Prot. Ver.} & \textbf{Data Length} & \textbf{Data} & \textbf{CRC} & \textbf{Postscript}\\
        \hline
        \multirow{2}{*}{1st Byte} & \multirow{2}{*}{2nd Byte} & \multirow{2}{*}{3rd Byte} & next Bytes & \multirow{2}{*}{Penultimate Byte} & \multirow{2}{*}{Final Byte} \\ 
        & & & up to 251 Bytes & & \\ \hline
    \end{tabular}
\end{table}


The first segment is the start byte, or the preamble. The preamble is followed by the packet sequence number and protocol
version, each of which are 4 bits long and are together packed into a single byte.
Next, one byte field that reports the length of the data which follows it
immediately. The data segment is followed by a single CRC byte, and finally the packet ends with a one byte
postscript. Table \ref{table:seg} lists the packet and the static values, if any, for each of the segments.
\\


\begin{table}[H]
    \centering
    {
    \begin{tabular}{|c|c|c|c|}
        \hline
        \rowcolor{black!8}
        \textbf{Field} & \textbf{Value} & \textbf{Segment} & \textbf{Length}\\
        \hline
        Preamble & 0xAA & 1 & 1 Byte\\ \hline
        Packet Sequence Number & Variable & \multirow{2}{*}{2} & 1 Nibble\\ \cline{1-2} \cline{4-4}
        Protocol version & 0x00 &  & 1 Nibble\\ \hline
        Length of data (not whole packet) & Variable & 3 & 1 Byte\\ \hline
        Data & Variable & 4 & Variable \\ \hline
        CRC of data (not whole packet) & Variable & 5 & 1 Byte\\ \hline
        Postscript & 0x55 & 6 & 1 Byte\\ \hline
    \end{tabular}
    }
    \caption{Transmission Packet Segments}
    \label{table:seg}
\end{table}


\subsection{Data Sub-packets} \label{ssec:sub-pack}

The data segment of the transmission packet is further separated into many
sub-packets. Table \ref{table:packsegments} below shows the organization
of a sub-packet.
The sub-packet starts with a source identifier. One bit
validity field and seven bits ``length of the sub-packet'' field
are packed together as the next byte. The length field counts the number of
bytes following it which make up the sub-packet. 
The validity bit is set to 1 if the sensor reading is valid and set to 0 if the sensor
is dead, disabled, unconnected, unresponsive or if data could not be collected
from the sensor in the time window. The size of the sub-packet is restricted to 127 Bytes by the seven bits length field.
\par
The packet validity is initially 0, and it will be changed to 1 when each sub-packet gets sensor value from sensor. 
And after a transmission packet is trasmitted, the validity becomes 0 again. 
When validity is set to 0, if the sensor is dead, disabled, unconnected, unresponsive or data could not be collected,
the particular invalid sub-packet is not packed into a transmission packet.
\\

\begin{table}[H]
    \centering
    {
    \begin{tabular}{|c|c|c|}
        \hline
        \rowcolor{black!8}
        \textbf{Source ID} & \bf{1-bit Validity [0: invalid, 1: valid]| 7-bits Data Length} & \textbf{Data} \\ \hline
        1 Byte & 1 Byte & up to 127 Bytes \\
        \hline
    \end{tabular}
    }
    \caption{Transmission Packet Segments}
    \label{table:packsegments}
\end{table}




\subsection{Data Packer CRC} \label{ssec:crc-calc}

To validate the data transmitted from the sensor board, a CRC value for the data is
calculated and transmitted as part of the data packet. The Maxim 1-Wire
CRC polynomial is used for calculating the CRC.  On receiving the data packet, the CRC
of the data packet is recalculated and compared with the value transmitted as part of
the packet. If the two CRC values match, the transmission is error-free.
The equivalent polynomial function of the CRC is shown in Equation \ref{eq:CRC}.

\begin{equation}
\label{eq:CRC}
CRC = x^8 + x^5 + x^4 + 1
\end{equation}

Further description of the Maxim 1-Wire CRC is available in Maxim Application Note 27. Below are
the Python and C implementations of the CRC calculator. The CRC implementations below take a
data Byte and the previous CRC as inputs, and return the new CRC as return value.
\\

\textbf{Python Code:}
\begin{mdframed}
\begin{lstlisting}
def calc_crc (data_Byte,CRC_Value)
    CRC_Value = ord(data_Byte) ^ CRC_Value
    for j in range(8):
    if (CRC_Value  & 0x01):
        CRC_Value  = (CRC_Value  >> 0x01) ^ 0x8C
    else:
        CRC_Value  =  CRC_Value  >> 0x01
return CRC_Value
\end{lstlisting}
\end{mdframed}

\vskip 0.1in
\textbf{C Code:}
\begin{mdframed}
\begin{lstlisting}
unsigned char  CRC_CALC (unsigned char data, unsigned char crc) 
{ 
        unsigned char i;
        crc ^= data;
        for (i=0x00; i < 0x08; i++)
        {
                if (crc & 0x01) { crc = (crc >> 0x01)^0x8C; }
                else { crc =  crc >> 0x01; }
        }
        return(crc);
}
\end{lstlisting}
\end{mdframed}