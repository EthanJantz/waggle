\newpage
\section{Sub-packets}

As shortly explained in document section \ref{ssec:sub-pack}, data sub-packets are generated depending on its designated data format and length when data reading from each sensor if valid. The first byte of the sub-packet is sensor ID for each parameter, and the second byte means validity of the packet and length of the sensor data as shown in Table \ref{table:packsegments}. Detail of sub-packet and sensor data will be explined in this section.


\subsection{Parameters}

The sensor boards output a set of parameters which are identified by a unique ID. Each parameter
has a set of values associated with it which are encoded in an appropriate data format. The table
below lists the various parameters produced by the sensor boards, the unique source ID used to identify them, the values produced by them and the format in which the value is encoded.
\par
Each parameter and its values are composed into a sub-packet based on
the format described in document section \ref{ssec:sub-pack}.
In the case of parameters with 2 or more values, the encoded values are
arranged in the sub-packets sequentially. 


\begin{center}
\begin{longtable}{|l|c|>{\centering}p{0.3\textwidth}|l|}
\caption{Data sub-packet structure (each row is a "chunk")} \label{tab:dataChunk} \\

\hline \rowcolor{black!8} \multicolumn{1}{|c|}{\textbf{Parameter}} & \multicolumn{1}{c|}{\textbf{Source ID}} & \multicolumn{1}{c|}{\textbf{Values}} & \multicolumn{1}{c|}{\textbf{Formats}} \\ \hline
\endfirsthead

\multicolumn{4}{c}%
{{\bfseries \tablename \thetable{} -- continued from previous page}} \\
\hline \rowcolor{black!8} \multicolumn{1}{|c|}{\textbf{Parameter}} & \multicolumn{1}{c|}{\textbf{Source ID}} & \multicolumn{1}{c|}{\textbf{Values}} & \multicolumn{1}{c|}{\textbf{Formats}} \\ \hline 
\endhead

\rowcolor{black!8} \multicolumn{4}{|r|}{{Continued on next page}} \\ \hline
\endfoot

\hline
\endlastfoot

    \rowcolor{black!5} \multicolumn{4}{|c|}{{Airsense board}} \\ \hline
        Airsense/Lightsense MAC address & 0x00 & MAC Address & Format 3 \\ \hline
        TMP112 & 0x01 & Temperature & Format 6\\ \hline
        \multirow{2}{*}{HTU21D} & \multirow{2}{*}{0x02} & Temperature & \multirow{2}{*}{Format 6}\\ \cline{3-3}
        & & relative humidity & \\ \hline
        \multirow{2}{*}{BMP180} & \multirow{2}{*}{0x04} & Temperature & Format 6\\ \cline{3-4}
        & & Pressure & Format 4 \\ \hline
        PR103J2 & 0x05 & Temperature & Format 1\\ \hline
        TSL250RD & 0x06 & Visible Light & Format 1\\ \hline
        \multirow{4}{*}{MMA8452Q} & \multirow{4}{*}{0x07} & Acceleration in X & \multirow{4}{*}{Format 6}\\ \cline{3-3}
        & & Acceleration in Y & \\ \cline{3-3}
        & & Acceleration in Z & \\ \cline{3-3}
        & & Vibration & \\ \hline
        SPV1840LR5H-B & 0x08 & RMS Sound Level & Format 1\\ \hline
        TSYS01 & 0x09 & Temperature & Format 6\\ \hline

    \rowcolor{black!8} \multicolumn{4}{|c|}{{Lightsense board}} \\ \hline
        \multirow{3}{*}{HMC5883L} & \multirow{3}{*}{0x0A} & Magnetic Field in Z & \multirow{3}{*}{Format 8}\\ \cline{3-3}
        & & Magnetic Field in Y & \\ \cline{3-3}
        & & Magnetic Field in Z & \\ \hline
        \multirow{2}{*}{HIH6130} & \multirow{2}{*}{0x0B} & Temperature & \multirow{2}{*}{Format 6}\\ \cline{3-3}
        & & relative humidity & \\ \hline
        APDS-9006-020 & 0x0C & Ambient light intensity & Format 1\\ \hline
        TSL260RD & 0x0D & IR intensity & Format 1\\ \hline
        TSL250RD & 0x0E & Visible light intensity & Format 1\\ \hline
        MLX75305 & 0x0F & Light & Format 1\\ \hline 
        ML8511 & 0x10 & UV intensity & Format 1\\ \hline
        TMP421 & 0x13 & Temperature & Format 6\\ \hline
        SPV1840LR5H-B & 0x14 & RMS Sound Level & Format 1\\ \hline

    \rowcolor{black!8} \multicolumn{4}{|c|}{{Chemsense board}} \\ \hline
        Total reducing gases & 0x15 & \multirow{7}{*}{Raw Concentration} & \multirow{7}{*}{Format 5}\\ \cline{1-2}
        Nitrogen dioxide & 0x17 & & \\ \cline{1-2}
        Ozone & 0x18 & & \\ \cline{1-2}
        Hydrogen sulphide & 0x19 & &\\ \cline{1-2}
        Total oxidizing gases & 0x1A & &\\ \cline{1-2}
        Carbon monoxide & 0x1B & &\\ \cline{1-2}
        Sulfur dioxide & 0x1C & &\\ \hline
        \multirow{2}{*}{SHT25} & \multirow{2}{*}{0x1D} & Temperature & \multirow{2}{*}{Format 2}\\ \cline{3-3}
        & & relative humidity & \\ \hline
        \multirow{2}{*}{LPS25H} & \multirow{2}{*}{0x1E} & Temperature & Format 2\\ \cline{3-4}
        & & Pressure & Format 4\\ \hline
        \multirow{3}{*}{Si1145} & \multirow{3}{*}{0x1F} & UV intensity & \multirow{3}{*}{Format 1} \\ \cline{3-3}
        & & Visible light intensity & \\ \cline{3-3}
        & & IR intensity & \\ \hline
        Chemsense MAC address & 0x20 & MAC Address & Format 3\\ \hline
        CO ADC temp & 0x21 & \multirow{3}{*}{ADC temperature} &  \multirow{3}{*}{Format 2}\\ \cline{1-2}
        IAQ IRR ADC temp & 0x22 & &\\ \cline{1-2}
        O3 NO2 ADC temp & 0x23 & &\\ \hline
        SO2 H2S ADC temp & 0x24 & \multirow{2}{*}{ADC temperature} &  \multirow{2}{*}{Format 2}\\ \cline{1-2}
        CO LMP temp & 0x25 & &\\ \hline
        \multirow{4}{*}{Accelerometer} & \multirow{4}{*}{0x26} & Acceleration in X & \multirow{3}{*}{Format 2} \\ \cline{3-3}
        & & Acceleration in Y & \\ \cline{3-3}
        & & Acceleration in Z & \\ \cline{3-4}
        & & Vibration & Format 4\\ \hline
        \multirow{4}{*}{Gyro} & \multirow{4}{*}{0x27} & Orientation in X & \multirow{3}{*}{Format 2} \\ \cline{3-3}
        & & Orientation in Y & \\ \cline{3-3}
        & & Orientation in Z & \\ \cline {3-4}
        & & Orientation Index & Format 4\\
\end{longtable}
\end{center}

\subsection{Data packets}
The context of each parameter, its utility and the arrangement of its values is described below. In all
the tables below, the validity bit is set to 1, which means the data is valid. The parameter descriptions
below are aggregated based on the sensor-board they are situated on -
Metsense, Lightsense and Chemsense.

\subsubsection{Metsense}

\paragraph{$\bullet$ Metsense/Lightsense MAC address: }

This is a six byte ID that uniquely identifies each Airsense board. This MAC address is also applied to each Lightsense board which has the same board number. The ID is provided by a DS2401 1-Wire DSN chip. The 1-byte family ID and CRC provided by the DSN chip are omitted, and the rest six bytes are used as the Unique ID. The Unique ID uses Format 3 for encoding and the arrangement is listed below.
\\

\begin{table}[h!]
    \centering
    \caption{Sub-packet of Met/lightsense board MAC address}
    \begin{tabular}{|c|c|c|}
        \hline
        \rowcolor{black!8}
        \textbf{0x00} & \textbf{0x86} & \textbf{MAC address in Format 3} \\
        \hline
        Byte[0] & Byte[1] & Bytes[2 -- 7]\\ \hline
    \end{tabular}
\end{table}
\par

\paragraph{$\bullet$ TMP112, TSYS01:}

TMP112 and TSYS01 are digital temperature sensors, which provides the temperature values
in centigrade.

\begin{table}[h!]
    \centering
    \caption{Sub-packet of a temperature sensor, TMP112}
    \begin{tabular}{|c|c|c|}
        \hline
        \rowcolor{black!8}
        \textbf{Sensor ID} (0x01 or 0x09) & \textbf{0x82} & \textbf{Temperature in Format 6} \\
        \hline
        Byte[0] & Byte[1] & Bytes[2 -- 3]\\ \hline
    \end{tabular}
\end{table}


\paragraph{$\bullet$ HTU21D:}
HTU21D is a digital temperature and relative humidity sensor, which provides
relative humidity value (\%RH) and temperature value in centigrade.

\begin{table}[h!]
    \centering
    \caption{Sub-packet of a temperature and relative humidity sensor, HTU21D}
    \begin{tabular}{|c|c|c|c|}
        \hline
        \rowcolor{black!8}
        \textbf{0x02} & \textbf{0x84} & \textbf{Temperature in Format 6} & \textbf{Relative humidity in Format 6}\\
        \hline
        Byte[0] & Byte[1] & Bytes[2 -- 3] & Bytes[4 -- 5] \\ \hline
    \end{tabular}
\end{table}


\paragraph{$\bullet$ BMP180:}

BMP180 is an digital temperature and barometric pressure sensor,
which provides temperature in centigrade and pressure in hectopascals.


\begin{table}[h!]
    \centering
    \caption{Sub-packet of a temperature and barometric pressure sensor, BMP180}
    \begin{tabular}{|c|c|c|c|}
        \hline
        \rowcolor{black!8}
        \textbf{0x04} & \textbf{0x84} & \textbf{Temperature in Format 6} & \textbf{Barometric pressure in Format 4}\\
        \hline
        Byte[0] & Byte[1] & Bytes[2 -- 3] & Bytes[4 -- 5] \\ \hline
    \end{tabular}
\end{table}


\paragraph{$\bullet$ PR103J2:}

PR103J2 is an analog temperature sensor whose resistance changes with change in temperature.
The sensor is implemented in a voltage divider circuit, and the center-tap voltage is converted and packed into Format 1 using a 10-bit ADC.


\begin{table}[h!]
    \centering
    \caption{Sub-packet of a temperature sensor, PR103J2}
    \begin{tabular}{|c|c|c|}
        \hline
        \rowcolor{black!8}
        \textbf{0x05} & \textbf{0x82} & \textbf{Voltage output in Format 1} \\
        \hline
        Byte[0] & Byte[1] & Bytes[2 -- 3]\\ \hline
    \end{tabular}
\end{table}


\paragraph{$\bullet$ TSL250RD:}

TSL250RD is an analog visible light sensor that produces an analog voltage that is
representative of the irradiance measured in $\mu$W/cm$^2$. The output voltage of the sensor
is converted and packed into Format 1 using a 10-bit ADC.


\begin{table}[h!]
    \centering
    \caption{Sub-packet of a light intensity sensor, TSL250}
    \begin{tabular}{|c|c|c|}
        \hline
        \rowcolor{black!8}
        \textbf{0x06} & \textbf{0x82} & \textbf{Voltage output in Format 1} \\
        \hline
        Byte[0] & Byte[1] & Bytes[2 -- 3]\\ \hline
    \end{tabular}
\end{table}


\paragraph{$\bullet$ MMA8452Q:}

MMA8452Q is a digital three-axis accelerometer. The accelerations in three orthogonal directions,
x, y and z, as a multiple of acceleration due to gravity (g) are obtained from the sensor,
and a vibration value (represented as multiple of g) is calculated using high-frequency
time series data from the three axis.

\begin{table}[h!]
    \centering
    \caption{Sub-packet of a three-axis accelerometer, MMA8452Q}
    \begin{tabular}{|c|c|c|c|c|c|}
        \hline
        \rowcolor{black!8}
        \textbf{0x07} & \textbf{0x88} & Acc(x) in Format 6 & Acc(y) in Format 6 & Acc(z) in Format 6 & Vib. in Format 6\\
        \hline
        Byte[0] & Byte[1] & Bytes[2 -- 3] & Bytes[4 -- 5] & Bytes[6 -- 7] & Bytes[8 -- 9] \\ \hline
    \end{tabular}
\end{table}



\paragraph{$\bullet$ SPV1840LR5H-B:}

SPV1840LR5H is a MEMS microphone that is sampled at high frequency to obtain
the peaks and calculate the sound intensity for a time window. The raw calculated
intensity is represented as a 16-bit integer value using Format 1.


\begin{table}[h!]
    \centering
    \caption{Sub-packet of a sound level sensor, SPV1840LR5H-B}
    \begin{tabular}{|c|c|c|}
        \hline
        \rowcolor{black!8}
        \textbf{0x08} & \textbf{0x82} & \textbf{Voltage output in Format 1} \\
        \hline
        Byte[0] & Byte[1] & Bytes[2 -- 3]\\ \hline
    \end{tabular}
\end{table}


\subsubsection{Lightsense}

\paragraph{$\bullet$ HMC5883L:}
HMC5883L is a digital three-axis magnetometer. The magnetic field strengths in three orthogonal directions,
x, y and z are obtained from the sensor.


\begin{table}[h!]
    \centering
    \caption{Sub-packet of a three-axis magnetometer, HMC5883L}
    \begin{tabular}{|c|c|c|c|c|}
        \hline
        \rowcolor{black!8}
        \textbf{0x0A} & \textbf{0x86} & \textbf{Strength Hx in Format 8} & \textbf{Strength Hy in Format 8} & \textbf{Strength Hz in Format 8}\\
        \hline
        Byte[0] & Byte[1] & Bytes[2 -- 3] & Bytes[4 -- 5] & Bytes[6 -- 7] \\ \hline
    \end{tabular}
\end{table}



\paragraph{$\bullet$ HIH6130:}

HIH6130 is a digital temperature and relative humidity sensor, which provides
relative humidity value (\%RH) and temperature value in centigrade.


\begin{table}[h!]
    \centering
    \caption{Sub-packet of a temperature and relative humidity sensor, HIH6130}
    \begin{tabular}{|c|c|c|c|}
        \hline
        \rowcolor{black!8}
        \textbf{0x0B} & \textbf{0x84} & \textbf{Temperature in Format 6} & \textbf{Relative Humidity in Format 6}\\
        \hline
        Byte[0] & Byte[1] & Bytes[2 -- 3] & Bytes[4 -- 5] \\ \hline
    \end{tabular}
\end{table}


\paragraph{$\bullet$ APDS-9006-020, TSL260, TSL250, MLX75305, and ML8511:}

APDS-9006-020, TSL260, TSL250, MLX75305, and ML8511 are analog light intensity sensors that produce the analog voltage 
representing the general luminance or the irradiance measured in $\mu$W/cm$^2$. The output voltage of the sensor
is converted and packed into Format 1 using a 16-bit ADC.
\\\\

\begin{table}[h!]
    \centering
    \caption{Sub-packet of light intensity sensors, APDS-9006-020, TSL260, TSL250, MLX75305, and ML8511}
    \begin{tabular}{|c|c|c|c|}
        \hline
        \rowcolor{black!8}
        \textbf{Sensor ID} (0x0C $\sim$ 0x10) & \textbf{0x82} & \textbf{Voltage output in Format 1}\\
        \hline
        Byte[0] & Byte[1] & Bytes[2 -- 3] \\ \hline
    \end{tabular}
\end{table}



\paragraph{$\bullet$ TMP421:}
TMP421 is a digital temperature sensor, which provides the temperature values
in centigrade.

\begin{table}[h!]
    \centering
    \caption{Sub-packet of a temperature sensor, TMP421}
    \begin{tabular}{|c|c|c|}
        \hline
        \rowcolor{black!8}
        \textbf{0x13} & \textbf{0x82} & \textbf{Temperature in Format 6}\\
        \hline
        Byte[0] & Byte[1] & Bytes[2 -- 3] \\ \hline
    \end{tabular}
\end{table}


\paragraph{$\bullet$ SPV1840LR5H-B}

SPV1840LR5H is a MEMS microphone that is sampled at high frequency to obtain
the peaks and calculate the sound intensity for a time window. The raw calculated
intensity is represented as a 16-bit integer value using Format 1.

\begin{table}[h!]
    \centering
    \caption{Sub-packet of a sound level sensor, SPV1840LR5H-B}
    \begin{tabular}{|c|c|c|}
        \hline
        \rowcolor{black!8}
        \textbf{0x14} & \textbf{0x82} & \textbf{Voltage output in Format 1}\\
        \hline
        Byte[0] & Byte[1] & Bytes[2 -- 3] \\ \hline
    \end{tabular}
\end{table}


\subsubsection{Chemsense:}

\paragraph{$\bullet$ Chemical sensors -- Total reducing gases, Nitrogen dioxide, Ozone, Hydrogen sulphide, Total oxidizing gases, Carbon monoxide, and Sulfur dioxide:}
These parameters provide the current output of the electrochemical ToR, NO$_2$, O$_3$, H$_2$S, ToX, CO, and SO$_2$ sensor.
Each of the cell current is quantified using an AFE that uses a
24-bit ADC to convert it into a signed digital value. This value is
represented in Format 5.

\begin{table}[h!]
    \centering
    \caption{Sub-packet of a temperature sensor, TMP421}
    \begin{tabular}{|c|c|c|}
        \hline
        \rowcolor{black!8}
        \textbf{Sensor ID} (0x15 $\sim$ 0x1C) & \textbf{0x83} & \textbf{Raw concentartion in Format 5}\\
        \hline
        Byte[0] & Byte[1] & Bytes[2 -- 4] \\ \hline
    \end{tabular}
\end{table}


\paragraph{$\bullet$ SHT25:}

SHT25 is a temperature and relative humidity sensor. 100th of temperature and relative humidity are
encoded in Format 2.

\begin{table}[h!]
    \centering
    \caption{Sub-packet of a temperature and relative humidity sensor, SHT25}
    \begin{tabular}{|c|c|c|c|}
        \hline
        \rowcolor{black!8}
        \textbf{0x1D} & \textbf{0x84} & \textbf{100th of Temperature in Format 2} & \textbf{100th of Relative humidity in Format 2}\\
        \hline
        Byte[0] & Byte[1] & Bytes[2 -- 4] & Bytes[5 -- 6] \\ \hline
    \end{tabular}
\end{table}



\paragraph{$\bullet$ LPS25H:}

LPS25H is a temperature and barometric pressure sensor. 100th of temperature and barometric pressureare
encoded in Format 2 and Format 4 respectively.

\begin{table}[h!]
    \centering
    \caption{Sub-packet of a temperature and barometric pressure sensor, LPS25H}
    \begin{tabular}{|c|c|c|c|}
        \hline
        \rowcolor{black!8}
        \textbf{0x1E} & \textbf{0x85} & \textbf{100th of Temperature in Format 2} & \textbf{barometric pressure in Format 4}\\
        \hline
        Byte[0] & Byte[1] & Bytes[2 -- 4] & Bytes[5 -- 6] \\ \hline
    \end{tabular}
\end{table}


\paragraph{$\bullet$ Si1145:}
Si1145 is a light sensor for three factors; ultra-violet, visible, and infrared. 
The raw output from the sensor is encoded Format 1.

\begin{table}[h!]
    \centering
    \caption{Sub-packet of an ultra-violet sensor, Si1145}
    \begin{tabular}{|c|c|c|c|c|}
        \hline
        \rowcolor{black!8}
        \textbf{0x1F} & \textbf{0x86} & \textbf{Raw UV in Format 1} & \textbf{Raw visible light in Format 1} & \textbf{Raw IR in Format 1}\\
        \hline
        Byte[0] & Byte[1] & Bytes[2 -- 4] & Bytes[5 -- 6] & Bytes[7 -- 8] \\ \hline
    \end{tabular}
\end{table}


\paragraph{$\bullet$ Chemsense MAC address:}

This is a six byte ID that uniquely identifies each Chemsense board. The Unique ID uses Format 3
for encoding and the arrangement is listed below.


\begin{table}[h!]
    \centering
    \caption{Sub-packet of Chemsense MAC address}
    \begin{tabular}{|c|c|c|c|c|c|c|c|}
        \hline
        \rowcolor{black!8}
        \textbf{0x20} & \textbf{0x86} & \textbf{Chemsense board MAC address in Format 3}\\
        \hline
        Byte[0] & Byte[1] & Bytes[2 -- 7] \\ \hline
    \end{tabular}
\end{table}


\paragraph{$\bullet$ ADC Temperatures -- CO ADC Temp, IAQ/IRR ADC Temp, O3/NO2 ADC Temp, SO2/H2S ADC Temp, and CO LMT Temp:}
Chemsense board measures temperature of sensor ADCs. This includes five parameters and 
all of them give ADC temperature in 100ths of degree celsious. Format 2 is used for encoding and the arrangement is listed below. 


\begin{table}[h!]
    \centering
    \caption{Sub-packet of ADC Temperatures}
    \begin{tabular}{|c|c|c|}
        \hline
        \rowcolor{black!8}
        \textbf{Sensor ID} (0x21 $\sim$ 0x25) & \textbf{0x82} & \textbf{Temperature in Format 2}\\
        \hline
        Byte[0] & Byte[1] & Bytes[2 -- 3] \\ \hline
    \end{tabular}
\end{table}


\paragraph{$\bullet$ Accelerometer:}
The accelerations in three orthogonal directions, x, y and z, as a multiple of acceleration are obtained from the sensor, and a vibration index is calculated. 
Acceleration data are encoded in Format 2, and vibration index is encoded in Format 4. 
\\\\

\begin{table}[h!]
    \centering
    \caption{Sub-packet of Accelerometer}
    \begin{tabular}{|c|c|c|c|c|c|}
        \hline
        \rowcolor{black!8}
        \textbf{0x26} & \textbf{0x89} & \textbf{A(x) in Format 2} & \textbf{A(y) in Format 2} & \textbf{A(z) in Format 2}& \textbf{Vib. in Format 4}\\
        \hline
        Byte[0] & Byte[1] & Bytes[2 -- 3] & Bytes[4 -- 5] & Bytes[6 -- 7] & Bytes[8 -- 10] \\ \hline
    \end{tabular}
\end{table}


\paragraph{$\bullet$ Gyro:}
The gyro in three orthogonal directions, x, y and z, as a multiple of acceleration are obtained from the sensor, and a orientation index is calculated. 
Gyro data are encoded in Format 2, and orientation index is encoded in Format 4.

\begin{table}[h!]
    \centering
    \caption{Sub-packet of Gyro}
    \begin{tabular}{|c|c|c|c|c|c|}
        \hline
        \rowcolor{black!8}
        \textbf{0x27} & \textbf{0x89} & \textbf{O(x) in Format 2} & \textbf{O(y) in Format 2} & \textbf{O(z) in Format 2}& \textbf{Index in Format 4}\\
        \hline
        Byte[0] & Byte[1] & Bytes[2 -- 3] & Bytes[4 -- 5] & Bytes[6 -- 7] & Bytes[8 -- 10] \\ \hline
    \end{tabular}
\end{table}
