\newpage
\section{Data Formats}

The data sent in each sub-packet is encoded in one or more formats. Currently
we define eight formats for various types of data including integers, bytes,
and floating point numbers. The numerical range of these representations is
restricted to within the bounds of values that we expect from the various sensors
and other sources. Thus the encoding schemes are specifically designed to
effectively and efficiently encode the values expected in the sensor streams.
The eight formats, and the encoding schemes are listed below in Table \ref{table:overall}.
\\

\begin{table}[H]
    \centering
    {
    \begin{tabular}{|c|c|c|c|}
        \hline
        \rowcolor{black!8}
        \textbf{Format} & \textbf{Number of Bytes Used} & \textbf{Value Represented} & \textbf{Value Range} \\ \hline
        1 & 2 & unsigned int\_16 input & [0, 65535] \\ \hline
        2 & 2 & int\_16 input & [$-$32767, 32767] \\ \hline
        3 & 6 & byte input[6] & [0x00, 0xFFFFFFFFFFFF] \\ \hline
        4 & 3 & unsigned long\_24 input & [0, 16777215] \\ \hline
        5 & 3 & long\_24 input & [$-$8388607, 8388607] \\ \hline
        6 & 2 & float input & [$-$127.99, 127.99] \\ \hline
        7 & 4 & byte input[4] & [0x00, 0xFFFFFFFF] \\ \hline
        8 & 2 & float input & [$-$0 -- 31.999] \\ \hline
    \end{tabular}
    }
    \caption{Data formats}
    \label{table:overall}
\end{table}


\textbf{Format 1:}
\begin{table}[H]
    \centering
    {
    \begin{tabular}{|c|c|}
        \hline
        \rowcolor{black!8}
        \textbf{8 Data Bits} & \textbf{8 Least Significant Bits} \\ \hline
        Byte[0] & Byte[1] \\ \hline
    \end{tabular}
    }
    \caption{format 1}
    \label{table:format1}
\end{table}


\textbf{Format 2:}
\begin{table}[H]
    \centering
    {
    \begin{tabular}{|c|c|}
        \hline
        \rowcolor{black!8}
        \textbf{Sign Bit | 7 Most Significant Bits} & \textbf{8 Least Significant Bits} \\ \hline
        Byte[0] & Byte[1] \\ \hline
    \end{tabular}
    }
    \caption{format 2}
    \label{table:format2}
\end{table}


\textbf{Format 3:}
\begin{table}[H]
    \centering
    {
    \begin{tabular}{|c|c|c|c|c|c|}
        \hline
        \rowcolor{black!8}
        \textbf{Array[0]} & \textbf{Array[1]} & \textbf{Array[2]} & \textbf{Array[3]} & \textbf{Array[4]} & \textbf{Array[5]} \\ \hline
        Byte[0] & Byte[1] & Byte[2] & Byte[3] & Byte[4] & Byte[5] \\ \hline
    \end{tabular}
    }
    \caption{format 3}
    \label{table:format3}
\end{table}


\textbf{Format 4:}
\begin{table}[H]
    \centering
    {
    \begin{tabular}{|c|c|c|}
        \hline
        \rowcolor{black!8}
        \textbf{8 Most Significant Bits} & \textbf{Bits 15 -- 8} & \textbf{8 Least Significant Bits} \\ \hline
        Byte[0] & Byte[1] & Byte[2] \\ \hline
    \end{tabular}
    }
    \caption{format 4}
    \label{table:format4}
\end{table}


\textbf{Format 5:}
\begin{table}[H]
    \centering
    {
    \begin{tabular}{|c|c|c|}
        \hline
        \rowcolor{black!8}
        \textbf{Sign Bit | 7 Most Significant Bits} & \textbf{Bits 15 -- 8} & \textbf{8 Least Significant Bits} \\ \hline
        Byte[0] & Byte[1] & Byte[2] \\ \hline
    \end{tabular}
    }
    \caption{format 5}
    \label{table:format5}
\end{table}


\textbf{Format 6:} The leading bit of the Byte 1 is always set to 0
\begin{table}[H]
    \centering
    {
    \begin{tabular}{|c|c|c|}
        \hline
        \rowcolor{black!8}
        \textbf{Sign Bit| 7 bit representation of Integer part} & \textbf{ 0 | 7 bit representation of the Fractional part} \\ \hline
        Byte[0] & Byte[1] \\ \hline
    \end{tabular}
    }
    \caption{format 6}
    \label{table:format6}
\end{table}


\textbf{Format 7:}
\begin{table}[H]
    \centering
    {
    \begin{tabular}{|c|c|c|c|}
        \hline
        \rowcolor{black!8}
        \textbf{Array[0]} & \textbf{Array[1]} & \textbf{Array[2]} & \textbf{Array[3]} \\ \hline
        Byte[0] & Byte[1] & Byte[2] & Byte[3] \\ \hline
    \end{tabular}
    }
    \caption{format 7}
    \label{table:format7}
\end{table}


\textbf{Format 8:} Only 3 fractional places are allowed
\begin{table}[H]
    \centering
    {
    \begin{tabular}{|c|c|}
        \hline
        \rowcolor{black!8}
        \textbf{Sign Bit| 5 bit representation of Integer | 2 MSBs of fraction} & \textbf{8 LSBs of the fraction} \\ \hline
        Byte[0] & Byte[1] \\ \hline
    \end{tabular}
    }
    \caption{format 8}
    \label{table:format8}
\end{table}


\textbf{Sign bit:}
\newcolumntype{a}{>{\columncolor{black!8}}c}
\begin{table}[H]
    \centering
    {
    \begin{tabular}{|a|c|}
        \hline
        \textbf{Positive Number} & 0 \\ \hline
        \textbf{Negative Number} & 1 \\ \hline
    \end{tabular}
    }
    \caption{Sign bit}
    \label{table:signBit}
\end{table}
