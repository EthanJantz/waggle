\newpage
\section{Sensor Data Units}\label{section:parameterUnits}
\subsection { Raw and Processed} 


The sensor boards output a set of values which have various units for the data.
The table below lists the various units of sensor values.
`Raw Units' in the table means the unit of the packtized data, which is you can get directly from the packet, and `Processed Units' means the unit which can be used after data conversion through designated equations.
The equations will be provided comming subsections.


\begin{center}
\begin{longtable}{|l|c|c|c|}
\caption{Sensor units both in raw and processed format}
\label{table:parameterUnits} \\

\hline \rowcolor{black!8} \multicolumn{1}{|l|}{\textbf{Sensor/Parameter}} & \multicolumn{1}{c|}{\textbf{Raw Units}} & \multicolumn{1}{c|}{\textbf{Processed Units}} & \multicolumn{1}{c|}{\textbf{Comments}} \\ \hline
\endfirsthead

\multicolumn{4}{c}%
{{\bfseries \tablename\ \thetable{} -- continued from previous page}} \\

\hline \rowcolor{black!8} \multicolumn{1}{|l|}{\textbf{Sensor/Parameter}} & \multicolumn{1}{c|}{\textbf{Raw Units}} & \multicolumn{1}{c|}{\textbf{Processed Units}} & \multicolumn{1}{c|}{\textbf{Comments}} \\ \hline
\endhead

\rowcolor{black!8} \multicolumn{4}{|r|}{{Continued on next page}} \\ \hline
\endfoot

\hline
\endlastfoot
 
    Firmware version & No Units & & *See appendix A \\ \hline

  \rowcolor{black!5} \multicolumn{4}{|c|}{{Airsense board}} \\ \hline
    Air/Lightsense MAC & No Units & No Units & \\  \hline
    TMP112 & \degree C & \degree C & \\ \hline
    HTU21D & \degree C, \%RH & \degree C, \%RH & \\ \hline
    BMP180 & \degree C, Pa & \degree C, Pa & \\ \hline
    PR103J2 & integer & \degree C & \\ \hline
    TSL250RD & integer & $\mu$w/m$^2$ & \\  \hline
    MMA8452Q & g, g, g, g & g, g, g, g & \\ \hline
    SPV1840LR5H-B & integer & & \\ \hline
    TSYS01 & \degree C & \degree C & \\ \hline
    
  \rowcolor{black!5} \multicolumn{4}{|c|}{{Lightsense board}} \\ \hline
    HMC5883L & G, G, G & G, G, G & \\ \hline
    HIH6130 & \degree C, \%RH & \degree C, \%RH & \\ \hline
    APDS-9006-020 & integer & lux & \\  \hline
    TSL260RD & integer & $\mu$w/m$^2$ & \\  \hline
    TSL250RD & integer & $\mu$w/m$^2$ & \\  \hline
    MLX75305 & integer & $\mu$w/m$^2$ & \\  \hline
    ML8511 & integer & UV index & \\  \hline
    TMP421 & \degree C & \degree C & \\ \hline

  \rowcolor{black!5} \multicolumn{4}{|c|}{{Chemsense board}} \\ \hline
    Total reducing gases & \multirow{2}{*}{AFE ADC counts} & & \multirow{2}{*}{Raw ADC reading} \\  \cline{1-1}
    Nitrogen dioxide & & & \\  \hline
    Ozone & \multirow{5}{*}{AFE ADC counts} & & \multirow{5}{*}{Raw ADC reading} \\  \cline{1-1}
    Hydrogen sulphide & & & \\  \cline{1-1}
    Total oxidizing gases & & & \\  \cline{1-1}
    Carbon monoxide & & & \\  \cline{1-1}
    Sulfur dioxide & & & \\  \hline
    SHT25 & 100ths of \degree C / \%RH & \degree C, \%RH & \\ \hline
    LPS25H & 100ths of \degree C, Pa & \degree C, Pa & \\ \hline
    Si1145 & Three fixed dummy value & & Uncompleted FW\\ \hline
    Intel MAC address & No Units & No Units & \\  \hline
    CO ADC temp & \multirow{5}{*}{100ths of \degree C} & \multirow{5}{*}{\degree C} & \\ \cline{1-1} \cline{4-4}
    IAQ IRR ADC temp & & & \\ \cline{1-1} \cline{4-4}
    O3 NO2 ADC temp & & & \\ \cline{1-1} \cline{4-4}
    SO2 H2S ADC temp & & & \\ \cline{1-1} \cline{4-4}
    CO LMP temp & & & \\ \hline
    Accelerometer & \multirow{2}{*}{raw register} & & \multirow{2}{*}{Raw reading} \\ \cline{1-1}
    Gyro & & &\\ \hline

 \rowcolor{black!5} \multicolumn{4}{|c|}{{Alpha sensor}} \\ \hline
 \rowcolor{black!2} \multicolumn{4}{|c|}{{Histogram}} \\ \hline
    Bin count & raw integer & & \\ \hline
    Average time & raw integer & & value 10 = 3.33 $\mu$s \\ \hline
    Sample flow rate & ml/s & & \\ \hline
    Temp/Pressure(alter) & 10ths of $^{\circ}$C / Pa (alter) & & \\ \hline
    Sampling period & raw float & & \\ \hline
    Sum of the counts & raw integer & & \\ \hline
    PM1 & $\mu$g/m$^3$ & & \\ \hline
    PM2.5 & $\mu$g/m$^3$ & & \\ \hline
    PM10 & $\mu$g/m$^3$ & & \\ \hline
 
 \rowcolor{black!2} \multicolumn{4}{|c|}{{Serial Number}} \\ \hline
    Serial & raw register & & \\ \hline
 
 \rowcolor{black!2} \multicolumn{4}{|c|}{{Firmware}} \\ \hline
    Firmware & raw integer & & \\ \hline
 
 \rowcolor{black!2} \multicolumn{4}{|c|}{{Configuration}} \\ \hline
    \rowcolor{white} \multicolumn{4}{|c|}{{Configuration Packet A (Source ID 0x31)}} \\ \hline
    Bin boundaries & raw integer & & \\ \hline 
    Bin particle Volumes A & raw float & & \\ \hline
    
    \rowcolor{white} \multicolumn{4}{|c|}{{Configuration Packet B (Source ID 0x32)}} \\ \hline
    Bin particle Volumes B & \multirow{2}{*}{raw float} & & \\ \cline{1-1} \cline{3-4}
    Bin particle Densities A & & & \\ \hline

    \rowcolor{white} \multicolumn{4}{|c|}{{Configuration Packet C (Source ID 0x33)}} \\ \hline
    Bin particle Densities B & \multirow{2}{*}{raw float} & & \\ \cline{1-1} \cline{3-4}
    Bin sample Volume Weightings A & & & \\ \hline

    \rowcolor{white} \multicolumn{4}{|c|}{{Configuration Packet D (Source ID 0x34)}} \\ \hline
    Bin sample Volume Weightings B & \multirow{3}{*}{raw float} & & \\ \cline{1-1} \cline{3-4}
    Gain scaling Coefficient & & & \\ \cline{1-1} \cline{3-4}
    Sample flow Rate & & & \\ \hline
    Laser DAC & \multirow{4}{*}{raw integer} & & \\ \cline{1-1} \cline{3-4}
    Fan DAC & & & \\ \cline{1-1} \cline{3-4}
    Conversion factor & & & \\ \cline{1-1} \cline{3-4}
    Spare bytes & & &\\

\end{longtable}
\end{center}


\subsection{conversion processure}
\subsubsection{Airsense:}
\paragraph{$\bullet$ TMP112, HTU21D, BMP180, MMA8452Q, TSYS01:} \label{ssec:first}
Raw outputs from the sensor boards for the sensors (TMP112, HTU21D, HIH4030, BMP180, MMA8452Q, and TSYS01) are the designated type of sensor value.

\paragraph{$\bullet$ PR103J2:}
Output of PR103J2 is an interger indicating output voltage from the sensor, which is mapped into integer values between 0 and 1023 with voltage range 0 to 3.3V. The raw integer value can be converted to resistance value through the equations below. The resistance value is needed to find corresponding temperature in a resistance-temperature look-up table (PR103J2 R-T table).

{\centering
 \[ \text{resistance } (\Omega) = 47000 \times \left(\frac{1023}{\text{raw integer}} - 1\right) \]
 \par
 }

\paragraph{$\bullet$ TSL250RD:}

Output of TSL250RD in airsense board is an interger indicating output voltage from the sensor, which is mapped into integer values between 0 and 1023 with voltage range 0 to 3.3V. The raw interger value can be converted to irradiance of visible light in micro-watt per square meter through equations below.
% The equation refers to irradiance responsivity of TSL250RD which is 0.064 \(mV/(\mu W/cm^2\)), and 0.09 is output voltage of dark condition which is initial offset (without any light -- NEED TO BE CHECKED).


{\centering
\[ \text{irradiance } (\mu W/m^2) = \frac{\text{raw integer} \times 3.3}{1023} \times \frac{1}{0.064} \]
\par
}


\paragraph{$\bullet$ SPV1840LR5H-B:}

Output value of SPV1840LR5H-B is an interger indicating amplified output voltage from the sensor, which is mapped into integer values between 0 and 1023.
The raw output need to be converted to sound level in decibel (dB).
 % The raw output can be converted to sound level in decibel (dB) through equations below.
% The equation refers to external gain as 453.33 which is determined by electric circle design, and input reference voltage as 3.3 (see schematics v3.1). 
% The equation to calculate sound level is convensional equation for voltage output to decibel (dB).

% \bigbreak

% {\centering
%  \[ \text{output voltage }(V) = \frac{\text{raw integer} \times 3.3 - 1.75 \times 1023 \times 454.33}{453.33 \times 1023} \] \\
%  \[ \text{sound level } (dB) = -20 \times \log_{10} \left( \frac{\text{output voltage}}{3.3}\right) \]
%  \par
%  }


\bigbreak
\subsubsection{Lightsense}

\paragraph{$\bullet$ HMC5883L, HIH6130, and TMP421:}
Raw outputs from the sensor boards for the sensors (HMC5883L, HIH6130, and TMP421) are the designated sensor value.

\paragraph{$\bullet$ Light sensors using MCP3426 (Multiplexer) -- APDS-9006-020, TSL260RD, TSL250RD, MLX75305, ML8511:}
Packetized data of the ligth sensors (APDS-9006-020, TSL260RD, TSL250RD, MLX75305, and ML8511) are raw integer proportional to the output voltage from the sensor.  The raw integers can be converted to irradiance through equations below. 
\newline \newline
All the sensor data coming through a common multiplexer and voltage divider, to the voltage output from the sensor is needed to calculate as shown below.

\bigbreak

{\centering
 \[ \text{output voltage }(V) =  \text{output voltage} \times 0.0000625 \times \frac{5}{2} \]
 \par
 }

\begin{itemize}
\item[$\circ$] APDS-9006-020

Raw output value of APDS-9006-020 is an analog voltage which is proportional to the irradiance. The output voltage can be converted irradiance in lux through the equation below.

\bigbreak

{\centering
 \[ \text{irradiance } (\text{lux}) = \frac{\text{output voltage}}{0.001944} \]
 \par
 }

 
\item[$\circ$] TSL260RD

Raw output value of TSL260RD is an analog voltage which is inverse proportional to the irradiance. The output voltage can be calculated though the equation below.
Dark voltage is the output voltage at dark condition, and it is an unique parameter of each sensor, so that the dark voltage can be changed for individual sensor.

\bigbreak

{\centering
 \[ \text{irradiance } (\mu W/m^2) = \frac{\text{output voltage} - \text{dark voltage}}{0.058} \]
 \par
 }


\item[$\circ$] TSL250RD

Raw output value of TSL250RD is an analog voltage which is inverse proportional to the irradiance. The output voltage can be calculated though the equation below.
Dark voltage is the output voltage at dark condition, and it is an unique parameter of each sensor, so that the dark voltage can be changed for individual sensor.

\bigbreak

{\centering
 \[ \text{irradiance } (\mu W/m^2) = \frac{\text{output voltage} - \text{dark voltage}}{0.064} \]
 \par
 }
 

\item[$\circ$] MLX75305

Raw output value of MLX75305 is an analog voltage which is inverse proportional to the irradiance. The output voltage can be calculated though the equation below.
Dark voltage is the output voltage at dark condition, and it is an unique parameter of each sensor, so that the dark voltage can be changed for individual sensor.

\bigbreak

{\centering
 \[ \text{irradiance } (\mu W/m^2) = \frac{\text{output voltage} - \text{dark voltage}}{0.007} \]
 \par
 }
 

\item[$\circ$] ML8511

Raw output value of ML8511 is an analog voltage which is proportional to the irradiance. The output voltage can be calculated though the equation below.
Dark voltage, offset voltage, and UV error are unique parameters of each sensor, so that these values can be changed for individual sensor.
\\\\
Dark voltage is the output voltage at dark condition, offset voltage is difference voltage between output voltage at 10 mW/cm$^2$ and dark voltage,
and UV error is the error between real UV index and calculated UV index.

\bigbreak

{\centering
 \[ \text{UV index } = \left( \text{output voltage} - \text{dark voltage} \right) \times \frac{14.9916}{\text{offset voltage}} - \text{error term}\]
 \[ \text{error term} = \frac{14.9916}{\text{offset voltage}} - \text{UV error} \]
 \par
 }
 
\end{itemize}

\paragraph{$\circ$ SPV1840LR5H-B}

Raw output value of SPV1840LR5H-B is an analog voltage which is proportional to the sound level.
the Raw output need to be converted to sound level in decibel (dB).
 
% \bigbreak
% {\centering
% \[ \text{sound level } (dB) = -20 \times \log_{10} \left( \frac{\text{output voltage}}{3.3}\right) \]
% \par
% }


\subsubsection{Chemsense}

\paragraph{$\bullet$ Chemical sensors -- Total reducing gases, Nitrogen dioxide, Ozone, Hydrogen sulphide, Total oxidizing gases, Carbon monoxide, and Sulfur dioxide:}
AFE ADC values need to be conversed into ppm.


\paragraph{$\bullet$ SHT25, LPS25H:}
Given values of SHT25 and LPS25H are 100ths of temperature in Celsius and 100ths of relative humidity value.
If barometric pressure need to be converted in hPa, refer that hPa is 100 times of Pa.

{\centering
 \[ \text{temperature }(\degree C) = \frac{\text{output value}}{100} \]
 \[ \text{relative humidity }(\% RH) = \frac{\text{output value}}{100} \]
 \[ \text{barometric pressure }(hPa) = \frac{\text{output value}}{100} \]
 }

\paragraph{$\bullet$ Si1145:}

Si1145 is a light sensor. Raw values coming from the sensor are three fixed hex integers, however because Chemsense board driver is not completed the values are needed to be ignored.


\paragraph{$\bullet$ ADC Temperatures -- CO ADC Temp, IAQ/IRR ADC Temp, O3/NO2 ADC Temp, SO2/H2S ADC Temp, and CO CMT Temp:}
Chemsense board measures temperature of sensor ADCs. All of them give ADC temperature in 100ths of degree Celsius.


{\centering 
 \[ \text{temperature }(\degree C) = \frac{\text{output value}}{100} \]
 \par
}


\paragraph{$\bullet$ Accelerometer, Gyro:}
Raw reading of the sensor values need to be conversed into appropreate value.


\subsubsection{Alpha Sensor:}

\paragraph{$\bullet$ Histogram, Firmware, and Configuration}
Raw reading of the sensor values need to be conversed into appropreate value.