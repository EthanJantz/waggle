\section{Power Efficiency}

why and in what degree is power efficiency a requirement

Embedded systems have traditionally been the most power constrained computer systems. For deployment in urban areas these requirement is preserved but can be relaxed as it is easier to provide power on the system. In an urban environment provision can be made so that the system could switch to the town's electic distribution network when its battery or renewable resource are unsaficient. On the other hand, in rural areas power efficiency is much more important due to lack of an extensive power grid.

\subsection{Peripherals}

The choice of peripherals used has an impact on power consumption. As we show different usb-to-ethernet modules where consuming different amount of energy.

\begin{center}
 \begin{tabular}{| l | c |}
 \hline
   & Ampere \\
 \hline
 \hline
 Idle &  \\
 \hline
 + Ethernet1 & \\
 \hline
 + Ethernet2 & \\
 \hline
 + Keyboard & \\
 \hline
 \end{tabular}
\end{center}

Peripherals that are not necessary should be removed from the final deployment package (keyboard, UART)


\subsection{Local vs Remote processing}

In general the following can be assumed for cost of local versus remote processing of data.

\begin{center}
 \begin{tabular}{| l | c | c |}
 \hline
   & Processing Cost & Transfer Cost \\
 \hline
 \hline
 Local Processing & High & Low \\
 \hline
 Remote Processing & Low & High \\
 \hline
 \end{tabular}
\end{center}


A quick and low cost heuristic needs to be created in-order to decide if we will forward or process the raw sensor data.

\begin{lstlisting}
If (local_process_cost + processed_data_transfer_cost > raw_data_transfer_cost) {
  send(raw_data)
} else {
  process(raw_data)
  send(processed_data)
}
\end{lstlisting}

Another consideration especially when wireless communication is used, is the cost of dropped packages or retransmissions. In some cases the cost of retransmissions could surpass the power difference between local and remote processing of data. Thus the above algorithm should be augmented to represent that relation; possibly by multiplying the transfer cost with a retransmission factor.

\noindent
\textbf{Compression}

Use compression to reduce transmission cost. Can be beneficial under constraints.

\noindent
\textbf{Elastic Fidelity Computations}

Sensors are inherently faulty, thus imprecise computation can be employed to improve power efficiency


\subsection{Encryption}

Encryption is a process expensive process thus a power expensive one too. One approach to reduce power consumption of the encryption steps on the framework is to adopt symmetric encryption whenever possible. Also, different libraries could be profiled to find if there are significant benefits on choosing the one over the other.

\subsection{Kernel}

Kernel can be tuned to improve the power efficiency of the device. A first step would be to minimize the number of modules running at any point. Moreover, we can use a governor that has power-saving as a metric.

\textbf{WARNING: kernel 3.8 has a bug that causes reboot to fail when using the powersave governor. Use ``echo performance > /sys/devices/system/cpu/cpu0/cpufreq/scaling\_governor'' before reboot}


\begin{verbatim}
CPU Power Management  --->
 CPU Frequency scaling  --->
  Default CPUFreq governor (performance)  --->
   ( ) performance
   (X) powersave
   ( ) userspace
   ( ) ondemand
   ( ) conservative
\end{verbatim}


\subsection{The Buffer: buffer between the external and internal API}

The buffer in between the internal and external networks can be used to improve power efficiency. By choosing different policies for handling package forwarding we can reduce the consumed power. The reduced power comes at the cost of loss of accuracy.

The policies should be considered a second level techniques as initialy the polling frequency of the sensor should be altered; changing the polling frequency would provide even greater benefits. 

\subsubsection{Policies}

\begin{itemize}
 \item package drop
 \item sampling
 \item averaging
 \item compression
\end{itemize}

